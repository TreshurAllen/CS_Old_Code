% assignment5.m4  
%
% Treshur Allen-----Bruce Bolden-----CS121
% ------------MAY 6-------------GNU-------
%
%   An m4 file to package source files for submission.
%  After using m4 on this file, a single file will
%  be generated with the contents of the included files
%  (Hash.cpp, DictHash.cpp)
%
%  To generate a file for typesetting:
%
%   m4 assignment5.m4 > assignment5_TA.tex


\documentclass[12pt]{article}

%\usepackage[fancybox]
%\usepackage{color}  % old
\usepackage[usenames]{color}
\definecolor{light-gray}{gray}{0.65}

\begin{document}

\section{Introduction}

Here are the files for assignment 5, HashTables:

\pagebreak

\section{Program Log}

%  Text file

\begin{verbatim}
PROGRAM LOG

Time Required:
	about 10 hours actually coding 
	45 mins designing 
	1 hour actually implimenting 
Problems:
	The greatest problem I faced when starting this 
	assignment was trying to get the dict4.txt and dict8.txt
	files to read in properly. No matter how many different 
	things I tried, I would get a segmentation fault 
	everytime I ran the program. 
Different table sizes: 
	By changing the table size I found that it greatly 
	changed the placement of where the inserted values 
        were located at, this makes sense because the hash 
	value is based on the size of the table.


\end{verbatim}

\pagebreak

\section{Source Code}

%  Files

\subsection{Hash.cpp}

\begin{verbatim}

/* Hash.cpp
 *
 *  Hash table implementation from w/ header file: 
 *  Kernighan & Ritchie, The C Programming Language,
 *     Second Edition, Prentice-Hall, 1988.
 */

#include <iostream>
#include <iomanip>
#include <cstdlib>
#include <cstring>
#include <string>

using namespace std;

// hash header

struct nList     /*  table entry:  */
{
   char *word;          //  defined name of the word in the dictonary      
   int count = 0;
   struct nList *next;  //  next entry in chain  
};

typedef struct nList *NListPtr;

//-----------------------------------protos--------------------------
unsigned Hash( char *s );
NListPtr Lookup( char *s );
NListPtr Insert( char *word );
void PrintHashTable();
void PrintMinMax();

const int HASH_TABLE_SIZE = 8017;
static NListPtr hashTable[HASH_TABLE_SIZE];

//-------------------------------definitions----------------------
static char *Strdup( const char *s);  //  in cstring, but....



/*  Hash
 *  Generate hash value for string s
 */

unsigned Hash( char *s )
{
    unsigned hashVal;
    
    for( hashVal = 0 ; *s != '\0' ; s++ )
        hashVal = *s + 31 * hashVal;
        
    return  hashVal % HASH_TABLE_SIZE;
}


/*  Lookup
 *  Look for s in hashTable
 */

NListPtr Lookup( char *s )
{
    NListPtr np;
    
    for( np = hashTable[Hash(s)] ; np != NULL ; np = np->next )
    {
        if( strcmp(s, np->word) == 0 )
            return np;    //  found
    }
    
    return NULL;          //  not found
}

/*  Insert
 *  Put ( word ) in hash table
 */
 
NListPtr Insert( char *word )
{
    unsigned hashVal;
    NListPtr np;
    
    if( (np = Lookup(word)) == NULL )  // not found
    {
        np = (NListPtr) malloc(sizeof(*np));
        if( np == NULL || (np->word = Strdup(word)) == NULL )
            return NULL;
        hashVal = Hash(word);
        np->next = hashTable[hashVal];
        hashTable[hashVal] = np;
        np->count++;
    }
    else
	np->count++;
        
    return np;
}


/*  PrintHashTable
 *  Print the hash table contents
 */

void PrintHashTable()
{
    NListPtr np;
    cout << "--------------------\n" << endl;

    for( int i = 0 ; i < HASH_TABLE_SIZE ; i++ )
    {
        np = hashTable[i];
        while( np != NULL )
        {
	     cout << "Bucket    words" << endl;
             cout << setw(2) << i << ":    ";
             cout << np->count <<endl;
             np = np->next;
        }
    }
}

void PrintMinMax(){
  cout << "Here are the mins an maxs for each bucket:" << endl;
  cout << "------------------------------------------" << endl << endl;

  NListPtr np;

  for( int i = 0; i < HASH_TABLE_SIZE ; i++){
    np = hashTable[i];
    while( np != NULL ){
	cout << "Min  Max" << endl;
	cout << setw(2) << np->min << " , " << np->max;
	cout << endl;
	np = np->next;
    } // end while
  } // end for
}// end function

/*  Strdup
 *  Make a duplicate copy of s
 */

static char *Strdup( const char *s )
{
    char *p;
    
    p = (char * ) malloc(strlen(s)+1);          /*  +1 for '\0'  */
    if( p != NULL )
        strcpy(p,s);

    return p;
}



\end{verbatim}

\pagebreak

%  main file 

\subsection{DictHash.cpp}

\begin{verbatim}
/*  DictHash.cpp
 *  test file for assignment
*/

#include <iostream>
#include <cstdlib> 
#include <fstream>
#include <string>
using namespace std;

#include "Hash.cpp"

int main(){

 ifstream dict;
 dict.open( "dict8.h" );
 
 while( !dict.eof() ){
  char *Word;

  dict >> Insert( Word );
 } // end of while loop
 
 cout << "Here is the information about your dictonary table: " << endl;

 PrintHashTable();
 cout << endl;
 PrintMinMax();

} // end of main 

\end{verbatim}

\end{document}
